 \documentclass[c]{beamer}
%\documentclass{beamer}
\listfiles

\mode<presentation>
{
  %\usetheme[deutsch,titlepage0]{KIT}
\usetheme[deutsch]{KIT}
% \usetheme{KIT}

%%  \usefonttheme{structurebold}

  \setbeamercovered{transparent}

  \setbeamertemplate{enumerate items}[circle]
  %\setbeamertemplate{enumerate items}[ball]

}
\usepackage{babel}
\date{}
%\DateText

\newlength{\Ku}
\setlength{\Ku}{1.43375pt}

\usepackage[utf8]{inputenc}
\usepackage[TS1,T1]{fontenc}
\usepackage{array}
\usepackage{multicol}
\usepackage{lipsum}
\usepackage[]{algorithm2e}
\usepackage{amsmath}
\usepackage{color}

\usenavigationsymbols
%\usenavigationsymbols[sfHhdb]
%\usenavigationsymbols[sfhHb]

\subtitle{Algorithmen I SS 14}
\author[]{Vincent Schüßler}

\AuthorTitleSep{\relax}

\institute[ITI]{Institut für Theoretische Informatik}

\TitleImage[width=\titleimagewd]{images/title}

\newlength{\tmplen}

\newcommand{\verysmall}{\fontsize{6pt}{8.6pt}\selectfont}

\title[Algorithmen I SS 14]{Tutorium 13}

\usepackage{alltt}

\TitleImage[width=\titleimagewd]{images/title02}

\begin{document}

\begin{frame}
  \maketitle
\end{frame}

\begin{frame}
	\begin{center}
		\Huge
		Wiederholung
	\end{center}
\end{frame}

\begin{frame}{$\mathcal{O}$-Kalkül}
	Gegeben seien zwei Funktionen $f$ und $g$. Zeige dass gilt:

	$$ g \in \mathcal{O}(f) \Leftrightarrow f \in \Omega(g)$$
\end{frame}

\begin{frame}{Laufzeiten}
	Was ist der Unterschied zwischen Laufzeit, erwarteter Laufzeit und amortisierter Laufzeit?
\end{frame}

\begin{frame}{Sortieren}
	Gegeben sei eine Folge von $n$ Elementen (bestehend aus Schlüssel und Wert), wobei höchstens $k$ verschiedene Schlüssel auftreten.
	Entwickle einen Algorithmus, der die Folge in erwarteter Zeit $\mathcal{O}(k \log{k} + n)$ sortiert.
\end{frame}

\begin{frame}{Minimale Spannbäume}
	Sei $G = (V, E)$ ein zusammenhängender ungerichteter gewichteter Graph und $T$ ein MST in $G$.
	Für $V' \subseteq V$ sei $G'$ der von $V'$ induzierte Teilgraph von $G$ und $T'$ der von $V'$ induzierte Teilgraph von $T$.

	Zeige: Wenn $T'$ zusammenhängend ist, dann ist $T'$ ein MST in $G'$.
\end{frame}

\begin{frame}{Durchmesser von Graphen}
	Der Durchmesser eines Graphen ist der maximale Abstand zwischen zwei Knoten.
Entwickle einen Algorithmus, der den Durchmesser eines ungerichteten, ungewichteten, zusammenhängenden Graphen berechnet und dabei höchstens $\mathcal{O}(n m)$ Zeit benötigt.
\end{frame}

\begin{frame}{Geld wechseln (1)}
	Gegeben sei die Menge der Münzwerte $M = {1, 2, 5, 10, 20, 50, 100, 200}$.
	Entwickle einen Greedy-Algorithmus, der für einen gegebenen Geldbetrag die kleinste Menge an Münzen liefert, die zur Darstellung nötig sind.
	Funktioniert der Algorithmus noch, wenn zusätzlich eine Münze mit dem Wert $4$ eingeführt wird?
\end{frame}

\begin{frame}{Geld wechseln (2)}
	Gegeben sei eine beliebige Menge an Münzwerten $M$ mit $1 \in M$.
	Entwickle einen Algorithmus, der für einen gegebenen Geldbetrag die kleinste Menge an Münzen liefert, die zur Darstellung nötig sind.
\end{frame}

\end{document}
